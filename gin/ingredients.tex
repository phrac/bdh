\section{Botanicals Used in Gin Production} Gin producers have available to
them over 300 varying botanicals, each of which adds distinctive flavors and
fragrances to the finished product. The predominant botanicals should always be
juniper, followed closely by coriander and angelica root. Most distillers use
between six and twelve different botanicals to create their product and
differentiate it on the market. Flavors may become overly complex and lost when
mixed if more than ten botanicals are used. Likewise, less botanicals
accentuate the character of the gin without losing its complexity.

The folowing examines some of the more popular botanicals used and how they
contribute to the overall flavor and nose profile of gin.

\subsection*{Almond}\index{almond} The almond is native to South West Asia. Both
the bitter and sweet almond can be used in gin production. They are hard and so
have to be ground into a powder before distillation of the gin. It gives a
marzipan, nutty, spicy flavor to the gin.  

\subsection*{Angelica}\index{angelica} This is an aromatic root normally found
in the northern areas of France, Belgium, and Germany, and occasionally in
southern Norway. It is used as a fixative for the flavors and aromas imparted
by the other botanicals. It also gives a musky, nutty, woody, sweet flavor with
a piney, dry edge.  

\subsection*{Aniseed}\index{aniseed} Aniseed is the fragrant seed of the anise
plant (Pimpinella anisum), which has been used for centuries as a spice for
cooking and as a medicinal herb. It is a small plant found in North Africa and
Southern Europe which tastes like liquorice candy. It is also used to flavor
beverages other than gin, such as Anisette and Campari. The related star anise
is employed for such spirits as Sambuca. 

\subsection*{Calamus}\index{calamus} Calamus is an aromatic herb from Asia. It
is harvested by climbing palm trees. It was known to the ancient Greek
physician Hippocrates as a digestive tonic.  

\subsection*{Caraway}\index{caraway} This is a biennial European plant, a member
of the parsley family. Its fruits are small, spicy, aromatic seeds, widely used
in cooking and flavoring. These seeds have been found in archaeological digs in
Switzerland where they were dated as early as 6000 B.C.

\subsection*{Cardamom}\index{cardamoms} Cardamoms are seeds taken from plants,
members of the ginger family, which grow in India and China. Cardamoms are the
third most expensive spice in the world after vanilla and saffron.

\subsection*{Cassia}\index{cassia}\index{Chinese cinnamon|see{cassia}} Cassia
(Chinese cinnamon) is derived from any number of shrubs or trees belonging to
the Senna family. It is used as a flavoring agent, though the pulp of the seed
pods is useful as a laxative.  The dried buds of the plant are primarily
employed. Cassia is a member of the cinnamon family. It comes from the Acacia
tree that grows in Vietnam, China and Madagascar. It gives a cinnamon note to
the gin.

\subsection*{Cinnamon}\index{cinnamon} Cinnamon is the spice we all know which
adds heat and flavor to almost everything. It comes from the inner bark of a
number of varieties of the laurel tree, and in the preparation of gin is mainly
used as an undertone. 

\subsection*{Citrus Peel}\index{citrus peel} Citrus peels are used rather than
the flesh as they hold a larger portion of the flavorsome oils than the flesh.
The fruit comes from Spain and is dried before use in the gin. Lemon gives a
fresh, light, citrus note with the orange giving a bitter-sweet note.

\subsection*{Coriander}\index{coriander} Coriander is a parsley-like plant (the
fresh leaves are familiar as cilantro) whose pungent, strong-smelling seeds
have longitudinal ridges. It is one of the oldest known spices, and has long
been considered a medicinal herb which can strengthen the \emph{wind} or breath
by its beneficial effect upon the lungs and respiratory system. The seeds come
from Morocco, Romania, Moldavia and Bulgaria. The taste they give varies
according to where they come from. The seeds give a mellow, spicy, fragrant and
aromatic note with a gingery, lemon-sage.

\subsection*{Cubeb}\index{cubeb} These berries are the fruits of a shrub, a
member of the pepper family usually grown in Eastern India. They have been used
for centuries as an herb for the treatment of urinary problems and bronchial
    ailments. In the last century and even continuing in some quarters as late
    as the 1940s, they were smoked in the form of cigarettes. They come from
    Java and give a spicy, lemon-pine flavor to the gin. 

\subsection*{Cumin}\index{cumin} This spice is an annual of the carrot family
with fennel like leaves. It is used mainly in cooking and in Eastern; countries
as a condiment. It's also a vital ingredient in the favorite dish of the
South-Western United States -- chilli. 

\subsection*{Fennel}\index{fennel} Fennel is a tall stout herb of the parsley
family with yellow flowers whose seeds are highly pungent and used in many
cooking sauces. It serves as an aromatic fixative when used in gin. The plant
can grow quite large, sometimes reaching 15 feet in height. It is cultivated in
the U.S. as an herb for its seeds, and to use as a spice in cooking. 

\subsection*{Ginger}\index{ginger} The spice ginger comes from a root structure,
or rhizome, of the ginger plant. It is very commonly used in cooking and is
regarded as a general tonic.

\subsection*{Grains of Paradise}\index{grains of paradise} These are intensely
peppery berries from West Africa. The seeds of a plant \emph{Aframomum
Melegueta}, which is a member of the ginger family, they can be used to
intensify the flavoring effects of all the other botanicals in gin. These small
dark brown berries are also a member of the pepper family. They give a hot,
spicy, peppery flavor to the gin with hints of lavender, elderflower and mint.

\subsection*{Juniper}\index{juniper} These berries are small, hard, and
purplish-colored. The use of these berries, or the oil pressed from them,
imparts a piney, evergreen odor and taste; the smell sometimes even begins to
hint at turpentine. The juniper bush is indeed a member of the pine family and
has been known for centuries as a strong diuretic which has the affect of
cleaning out the kidneys. \textbf{By law this is the main flavor element in
gin.} Juniper comes from Italy and the former Yugoslavia, with the best ones
coming from Tuscany. They are handpicked between October and February. They
give a fragrant, spicy, bittersweet taste with overtones of pine, lavender and
camphor with a peppery finish.

\subsection*{Lemon}\index{lemon} The peel of this fruit is used to impart the
citrus astringency which gives gin its clean, dry nose and taste. Since only
the zest or colored portion of the peel is used as a gin flavoring agent, only
the oil from the peel is actually transferred. The best lemons are grown in
Italy and Spain.

\subsection*{Licorice}\index{licorice} is the root of a perennial herb found in
central and southern Europe. It's used in both medicine and candy-making, and
imparts the well-known piquant flavor, very similar to that which can also be
obtained from aniseed.

\subsection*{Nutmeg}\index{nutmeg} Nutmeg is an aromatic kernel of the fruit of
various tropical trees (genus Myristica), especially those of the nutmeg tree,
which imparts a musky flavor and aroma to gin.

\subsection*{Orange}\index{orange} The peel of both bitter and sweet oranges is
used, bitter to lend astringency in a manner similar to lemons, and sweet to
give an impression of sweetness. Since only the zest or colored part of the
peel is used, there is no transference of true sugars, only the
\emph{impression} of these.

\subsection*{Orris}\index{orris} Orris is the root of the Florentine iris. It
has a very perfumed flavor and helps fix the flavors in the gin. It is very
hard and requires heavy grinding into a powder before use. It has an aroma of
violets, earth and cold tea. It comes mainly from Florence in Italy, although
it is also grown in Peru and Morocco

\subsection*{Rosemary}\index{rosemary} Rosemary leaves are used as a spice and
come from the rosemary plant, an evergreen fragrant shrub \emph{Rosmarinus
officinalis} of the mint family. This plant grows in southern Europe and
western Asia and usually has small blue flowers. It has traditionally been
cultivated for its stimulating and refreshing aromas.

\subsection*{Savory}\index{savory} Savory is a hardy, annual, aromatic herb of
the mint family, used to bring out the flavors of the other herbs


\section{Botanicals Used in Gin Production}

\subsection{Almond}
The almond is native to South West Asia. Both the bitter and sweet almond can
be used in gin production. They are hard and so have to be ground into a powder
before distillation of the gin. It gives a marzipan, nutty, spicy flavour to
the gin.  

\subsection{Angelica}
This is an aromatic root normally found in the northern areas of France,
Belgium, and Germany, and occasionally in southern Norway. It is used as a
fixative for the flavours and aromas imparted by the other botanicals. It also
gives a musky, nutty, woody, sweet flavour with a piney, dry edge.  

\subsection{Aniseed}
Aniseed is the fragrant seed of the anise plant (Pimpinella anisum), which has
been used for centuries as a spice for cooking and as a medicinal herb. It is a
small plant found in North Africa and Southern Europe which tastes like
liquorice candy. It is also used to flavour beverages other than gin, such as
Anisette and Campari. The related star anise is employed for such spirits as
Sambuca 

\subsection{Calamus}
Aromatic herb from Asia, Calamus may be taken from climbing palm tree. It was
known to the ancient Greek physician Hippocrates as a digestive tonic.  

\subsection{Caraway}
This is a biennial European plant, a member of the parsley family. Its fruits
are small, spicy, aromatic seeds, widely used in cooking and flavouri ng. These
seeds have been found in archaeological digs in Switzerland where they were
dated as early as 6000 B.C

\subsection{Cardamom}
Cardamoms are seeds taken from plants, members of the ginger family, which
grow in India and China. Cardamoms are the third most expensive spice in the
world after vanilla and saffron.

\subsection{Cassia}
Cassia (Chinese cinnamon) is derived from any number of shrubs or trees
belonging to the Senna family. It is used as a flavouring agent, though th e
pulp of the seed pods is useful as a laxative.  Primarily the dried buds are
employed. Cassia is a member of the cinnamon family. It comes from the Acacia
tree that grows in Vietnam, China a nd Madagascar. It gives a cinnamon note to
the gin.

\subsection{Cinnamon}
Cinnamon is the spice we all know which adds heat and flavour to almost
everything. It comes from the inner bark of a number of varieties of the laurel
tree, and in the preparation of gin is mainly used as an undertone. Citrus Peel
– The peels are used rather than the flesh as they hold a larger portion of
the flavoursome oils than the fle sh. The fruit comes from Spain and is dried
before use in the gin. Lemon gives a fresh, light, citrus note with the orange
giving a bitter-sweet note.

\subsection{Coriander}
Coriander is a parsley-like plant (the fresh leaves are familiar as cilantro)
whose pungent, strong- smelling seeds have longitudinal ridges. It’s one o f
the oldest known spices, and has long been considered a medicinal herb which
can strengthen th e 'wind' or breath by its beneficial effect upon the lungs
and respiratory system The seeds com e from Morocco, Romania, Moldavia and
Bulgaria. The taste they give varies according to w here they come from. The
seeds give a mellow, spicy, fragrant and aromatic note with a gi ngery,
lemon-sage.

\subsection{Cubeb}
These berries are the fruits of a shrub, a member o f the "pepper family
usually grown in Eastern India. They have been used for centuries as an herb
for the treatment of urinary problems and bronchial ailments. In the last
century and even co ntinuing in some quarters as late as the 40s, they were
smoked in the form of cigarettes. They co me from Java and give a spicy,
lemon-pine flavour to the gin. 

\subsection{Cumin}
This spice is an annual of the carrot family with f ennel like leaves. It is
used mainly in cooking and in Eastern; countries as a condiment. It's also a
vital ingredient in the favourite dish of the South- Western United States-
chilli. 

\subsection{Fennel}
Fennel is a tall stout herb of the parsley family with yellow flowers whose
seeds are highly pungent and used in many cooking sauces. It serves as an
aromatic fixative when used in gin. The plant can grow quite large, sometimes
reaching 15 feet in height. It is cultivated in the U.S. as an herb for its
seeds, and to use as a spice in cooking. 



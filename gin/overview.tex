\index{Gin} 
\index{Genever} 
\index{Jenever|see{Genever}}
\index{Hollands|see{Genever}} 
Gin\footnote{The name gin is derived from either the French genièvre or the
Dutch jenever, both meaning ``juniper''.} is a spirit which derives its
predominant flavor from juniper berries \emph{Juniperus communis}. From its
earliest beginnings in the Middle Ages, gin has evolved over the course of a
millennium from a herbal medicine to an object of commerce in the spirits
industry. Today, the gin category is one of the most popular and widely
distributed of all spirits, and is represented by products of various origins,
styles, and flavor profiles that all revolve around the juniper berry as a
common ingredient. \parencite{wiki:gin}


%-----------------------
% Methods of Production
%-----------------------
\section{Methods of Production} There are several methods of gin production
available to the distiller. Each method has advantages and disadvantages. Most
large scale gin producers begin with a neutral spirit purchased from another
distiller. The neutral spirit is typically produced from cereal grains such as
wheat or barley, but other neutral spirits are available to produce different
characteristics.

%-----------------------
% Cold Compounding
%-----------------------
\index{cold compounding} 
\index{Gin!cold compounding} 
\subsection{Cold Compounding}
Cold compounding is a method of gin production commonly used for production of
cheaper (supermarket) \index{Gin!supermarket gin} gins.  Distillers or blenders
start with a base neutral spirit and oils and flavorings are added to give the
nose and taste of gin. Because these flavors are not distilled into the spirit,
they are easily lost when the bottle is opened. This style of gin was popular
in the American prohibition era and was often referred to as ``bathtub gin.''
\index{Gin!bathtub gin}

%-----------------------
% Distilled Gin
%-----------------------
\index{Gin!distilled}
\subsection{Distilled Gin}
Distilled gin is a method of gin production in which the botanicals are infused
into the spirit which is then cut with water to reach the desired alcohol by
volume (ABV\%). Several methods are used depending on the distiller's recipe
and economics. These methods include: 

%-----------------------
% Distilled Gin: One Shot
%-----------------------
\index{Gin!distilled!one shot method}
\subsubsection{One Shot Method}
With the one shot method, the distiller macerates the botanicals in neutral
spirit and water for a predetermined length of time, from a few hours to a few
days. The botanicals are then either left in the still charge or strained off
and it is distilled in a pot still. Water is then added to achieve the desired
ABV and the product is bottled.

%-----------------------
% Distilled Gin: Two Shot
%-----------------------
\index{Gin!distilled!two shot method}
\subsubsection{Two Shot Method}
Similar to the one shot method, the two shot method involves maceration of the
botanical bill, but in much higher quantitites. This maceration is then
distilled to create a more botanically concentrated product. It is then mixed
with neutral spirits to increase the volume. Finally, the product is cut with
water before bottling. The two shot method is more economical to produce
because it requires less operating time of the still.

%-----------------------
% Distilled Gin: Vapor Infused
%-----------------------
\index{Gin!distilled!vapor infused} 
\subsubsection{Vapor Infused Method} 
The vapor infusion method of gin production differs greatly from other methods.
Instead of macerating the botanicals with neutral spirits, the botanicals are
placed in a cage, often called a ``gin basket'', and placed in the vapor path
in the still. The alcohol vapors then pass through this gin basket,
extracting the flavors and aromas. The vapor infusion method can be used in
column or pot stills.


